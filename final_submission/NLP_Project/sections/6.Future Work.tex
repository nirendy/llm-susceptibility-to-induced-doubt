\section{Future Work}

Building on these findings, we propose several directions for future research and practical development to address the identified limitations and expand the scope of this study.

\textbf{Advanced Prompt Engineering}
Future work should focus on designing prompts that reduce positional bias and test the boundaries of prompt comprehension. Incorporating dynamic prompts that adapt to model-specific behaviors could further enhance robustness.


\textbf{Dataset Diversity}
Using datasets with varying complexity and domain specificity will help evaluate how contextual factors influence model performance and robustness.

\textbf{Doubt expressions}
In this work we did not find consistent reactions of LLMs to doubt, but we found other factors that influence model reactions and thus may interact with the effect of doubt. Controling for more such factors may lead to better understanding of models reaction to doubt. 
% \subsection{Practical Applications}

\textbf{Confidence Calibration}
Confidence metrics should be integrated into LLM systems to enable real-time evaluation and correction of responses. Developing tools to visualize and adjust confidence levels could aid applications in fields like education, healthcare, and customer service.

\textbf{Iterative Interaction Frameworks}
Future applications can leverage iterative questioning to enhance user-model interactions. 
